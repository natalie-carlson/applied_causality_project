\documentclass[12pt]{article}

% fonts
\usepackage[scaled=0.92]{helvet}   % set Helvetica as the sans-serif font
\renewcommand{\rmdefault}{ptm}     % set Times as the default text font

% dmb: not mandatory, but i recommend you use mtpro for math fonts.
% there is a free version called mtprolite.

% \usepackage[amssymbols,subscriptcorrection,slantedGreek,nofontinfo]{mtpro2}

\usepackage[T1]{fontenc}
\usepackage{amsmath}
\usepackage{amsfonts}

% page numbers
\usepackage{fancyhdr}
\fancypagestyle{newstyle}{
\fancyhf{} % clear all header and footer fields
\fancyfoot[R]{\vspace{0.1in} \small \thepage}
\renewcommand{\headrulewidth}{0pt}
\renewcommand{\footrulewidth}{0pt}}
\pagestyle{newstyle}

% geometry of the page
\usepackage[top=1in, bottom=1in, left=1in, right=1in]{geometry}

% paragraph spacing
\setlength{\parindent}{0pt}
\setlength{\parskip}{2ex plus 0.4ex minus 0.2ex}

% useful packages
\usepackage{natbib}
\usepackage{epsfig}
\usepackage{url}
\usepackage{bm}


\begin{document}

\begin{center}
  \Large \textbf{Week 10 Response: \textit{Song, Hao, \& Storey, 2015}} \\
  \vspace{0.1in}
  \normalsize Natalie Carlson \\
  \today
\end{center}

Having no baseline of genetics knowledge, I found this difficult to read and by nature very abstract. I tried to focus on the main contribution, which seemed to be captured in Figure 1 -- the idea that they could use the allele frequency to test for association between \textit{x} and \textit{y} via inverse regression. The allele frequency proxies for any latent variable \textit{z} which can be a confounder given the population structure (in my head I was imagining this referring to things like diet or climate). While this indeed seems useful, I was curious whether there are any potential extensions beyond the realm of genetics. If we think yes, I would love to see a more generalized representation of this model, because I got a bit bogged down in googling what SNPs and alleles are...

\end{document}
