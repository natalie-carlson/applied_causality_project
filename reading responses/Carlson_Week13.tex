\documentclass[12pt]{article}

\input{preamble}

\begin{document}

\begin{center}
  \Large \textbf{Week 13 Response: \textit{Schulam \& Saria, 2017}} \\
  \vspace{0.1in}
  \normalsize Natalie Carlson \\
  \today
\end{center}

This paper was certainly innovative, and I imagine the feature of being able to predict counterfactual continuous-time outcomes is quite useful for epidemiologists and others who study medical phenomena. I have to admit that as soon as I read the ``no unobserved confounders'' assumption I was considerably less motivated to work through the (confusing!) Gaussian Processes notation. 

I do take this paper as further evidence that the potential outcomes framework (loosely treated) is filtering its way into computer science and other disciplines that were previously less concerned with causality and counterfactual thinking. I recently read a paper in Information Systems Research whose entire purpose was to introduce potential outcomes to IS researchers, as recently as 2009 [1]. I am curious to see how this trend will affect the trajectory of research in various disciplines (imagining a counterfactual time series graph here, of course, \textit{a la} Schulam \& Saria).

[1] Mithas, Sunil, and Mayuram S. Krishnan. "From association to causation via a potential outcomes approach." Information Systems Research 20.2 (2009): 295-313.


\end{document}
