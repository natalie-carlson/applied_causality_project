\documentclass[12pt]{article}

% fonts
\usepackage[scaled=0.92]{helvet}   % set Helvetica as the sans-serif font
\renewcommand{\rmdefault}{ptm}     % set Times as the default text font

% dmb: not mandatory, but i recommend you use mtpro for math fonts.
% there is a free version called mtprolite.

% \usepackage[amssymbols,subscriptcorrection,slantedGreek,nofontinfo]{mtpro2}

\usepackage[T1]{fontenc}
\usepackage{amsmath}
\usepackage{amsfonts}

% page numbers
\usepackage{fancyhdr}
\fancypagestyle{newstyle}{
\fancyhf{} % clear all header and footer fields
\fancyfoot[R]{\vspace{0.1in} \small \thepage}
\renewcommand{\headrulewidth}{0pt}
\renewcommand{\footrulewidth}{0pt}}
\pagestyle{newstyle}

% geometry of the page
\usepackage[top=1in, bottom=1in, left=1in, right=1in]{geometry}

% paragraph spacing
\setlength{\parindent}{0pt}
\setlength{\parskip}{2ex plus 0.4ex minus 0.2ex}

% useful packages
\usepackage{natbib}
\usepackage{epsfig}
\usepackage{url}
\usepackage{bm}


\begin{document}

\begin{center}
  \Large \textbf{Week 5 Response: \textit{Imbens \& Rubin, Chapter 8}} \\
  \vspace{0.1in}
  \normalsize Natalie Carlson \\
  \today
\end{center}

This chapter was quite painstaking in its mathematics but surprisingly simple and appealing in its conceptual approach. It is intuitive that, as they note, the posterior distribution is not too sensitive to the choice of prior given a randomized experiment. However, observational data does not have this neat feature, as the assignment process is not random. I am curious to see how they approach this difficulty in later chapters, particularly since most of our class projects fall into this bucket (though I have lately been considering a lab experiment to bolster my causal argument).

I was intrigued by the introduction of $\rho$, the correlation coefficient between the two potential outcomes. Naturally the data cannot truly contain information about the relationship between the potential outcomes, as both outcomes cannot be observed, and so the posterior will always be equal to the prior. How then should we choose this prior? Is it always better to be conservative by assuming a higher correlation, or if not, how can domain knowledge inform the choice? It seems important to the estimation of the model.


\end{document}
