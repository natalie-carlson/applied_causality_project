\documentclass[12pt]{article}

\input{preamble}

\begin{document}

\begin{center}
  \Large \textbf{Week 8 Response: \textit{Morgan \& Winship, Chapter 9}} \\
  \vspace{0.1in}
  \normalsize Natalie Carlson \\
  \today
\end{center}

Instrumental variables are familiar territory for me, so I felt quite comfortable with this chapter. I'm glad Morgan \& Winship touched on the debate over their use in economics, because I share the view that the discipline has become a bit too obsessed with finding clever instruments at, perhaps, the expense of other interesting approaches. Oddly enough, though, I had never thought too deeply about the fact that use of IVs typically assumes that the estimated treatment effect is invariant across subpopulations. This seems quite obvious when put in the context of school vouchers -- of course taking advantage of a lottery only gives you the ATE on the subpopulation that complies with the lottery results -- but I've always run with the quick assumption that if you can find a natural application of random assignment, you're good to go.

I've seen Imbens \& Angrist's LATE framework before as well, which seems like a decent approach to this issue, cutesy terminology aside. Framing the estimation in this way makes clear the limitations of natural experiments of this type. There is no way to estimate the causal effect among the always-takers or never-takers (the \textit{defier} class is what I think always struck me as so silly about this approach, but then again, human psychology can be bizarre). Of course, if it is possible to find one and/or convince adminstrators of some program of interest to apply it, Heckman's ``perfect instrument'' and marginal treatment effect framework is quite appealing as well. 

\end{document}
