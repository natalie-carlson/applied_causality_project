\documentclass[12pt]{article}

% fonts
\usepackage[scaled=0.92]{helvet}   % set Helvetica as the sans-serif font
\renewcommand{\rmdefault}{ptm}     % set Times as the default text font

% dmb: not mandatory, but i recommend you use mtpro for math fonts.
% there is a free version called mtprolite.

% \usepackage[amssymbols,subscriptcorrection,slantedGreek,nofontinfo]{mtpro2}

\usepackage[T1]{fontenc}
\usepackage{amsmath}
\usepackage{amsfonts}

% page numbers
\usepackage{fancyhdr}
\fancypagestyle{newstyle}{
\fancyhf{} % clear all header and footer fields
\fancyfoot[R]{\vspace{0.1in} \small \thepage}
\renewcommand{\headrulewidth}{0pt}
\renewcommand{\footrulewidth}{0pt}}
\pagestyle{newstyle}

% geometry of the page
\usepackage[top=1in, bottom=1in, left=1in, right=1in]{geometry}

% paragraph spacing
\setlength{\parindent}{0pt}
\setlength{\parskip}{2ex plus 0.4ex minus 0.2ex}

% useful packages
\usepackage{natbib}
\usepackage{epsfig}
\usepackage{url}
\usepackage{bm}


\begin{document}

\begin{center}
  \Large \textbf{Week 7 Response: \textit{Pearl, Chapter 7}} \\
  \vspace{0.1in}
  \normalsize Natalie Carlson \\
  \today
\end{center}

This chapter was indeed a beast, and seemed to me to be a little disjointed at times, although perhaps I am just not seeing the big picture. What struck me is that the entire exercise seemed, largely, an attempt to replicate how humans process and infer causality. It's not clear to me, though, that humans are exceptionally good at this. We are good at processing localized physical causal relationships -- I push you, you fall over -- but frequently make causal attributions about more complex systems, particuarly social systems, that are not remotely correct. Human beings fail to take confounding variables into account all the time. Because of this, I like the precision focus -- ``principled minisurgeries'', my new favorite description for the \textit{do} operator -- to causality we have taken so far. After all, this is the whole point -- that we're terrible at this! 

I also thought the discussion of the potential outcomes framework was quick, and a little dismissive, and would love to discuss it further in class. Is there anything contributed by the Neyman-Rubin Framework that is not covered by Pearl's probabilistic structural model? 

\end{document}
