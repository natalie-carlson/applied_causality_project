\documentclass[12pt]{article}

% fonts
\usepackage[scaled=0.92]{helvet}   % set Helvetica as the sans-serif font
\renewcommand{\rmdefault}{ptm}     % set Times as the default text font

% dmb: not mandatory, but i recommend you use mtpro for math fonts.
% there is a free version called mtprolite.

% \usepackage[amssymbols,subscriptcorrection,slantedGreek,nofontinfo]{mtpro2}

\usepackage[T1]{fontenc}
\usepackage{amsmath}
\usepackage{amsfonts}

% page numbers
\usepackage{fancyhdr}
\fancypagestyle{newstyle}{
\fancyhf{} % clear all header and footer fields
\fancyfoot[R]{\vspace{0.1in} \small \thepage}
\renewcommand{\headrulewidth}{0pt}
\renewcommand{\footrulewidth}{0pt}}
\pagestyle{newstyle}

% geometry of the page
\usepackage[top=1in, bottom=1in, left=1in, right=1in]{geometry}

% paragraph spacing
\setlength{\parindent}{0pt}
\setlength{\parskip}{2ex plus 0.4ex minus 0.2ex}

% useful packages
\usepackage{natbib}
\usepackage{epsfig}
\usepackage{url}
\usepackage{bm}


\begin{document}

\begin{center}
  \Large \textbf{Week 12 Response: \textit{Gelmania}} \\
  \vspace{0.1in}
  \normalsize Natalie Carlson \\
  \today
\end{center}

I enjoyed both Gelman's papers and the discussion on Wednesday. Both were quite broad and non-technical, but I think what he provides is a useful framework for thinking about how to approach and frame causal questions, particularly for social scientists. Two elements were particularly useful for me: the distinction between forward and reverse causal questions, and the notion of the latter being a form of model checking. The recent faddishness of randomized program evaluations nonwithstanding, I think the most interesting questions in social science tend to be reverse causal questions, and they are interesting precisely \textit{because} our current mental model doesn't suffice to explain a particular phenomenon. The process of answering that question then becomes an iterative proess of rejecting that inadequate model, creating a new model in the form of a forward causal question, and then testing it. 


\end{document}
