\documentclass[12pt]{article}

% fonts
\usepackage[scaled=0.92]{helvet}   % set Helvetica as the sans-serif font
\renewcommand{\rmdefault}{ptm}     % set Times as the default text font

% dmb: not mandatory, but i recommend you use mtpro for math fonts.
% there is a free version called mtprolite.

% \usepackage[amssymbols,subscriptcorrection,slantedGreek,nofontinfo]{mtpro2}

\usepackage[T1]{fontenc}
\usepackage{amsmath}
\usepackage{amsfonts}

% page numbers
\usepackage{fancyhdr}
\fancypagestyle{newstyle}{
\fancyhf{} % clear all header and footer fields
\fancyfoot[R]{\vspace{0.1in} \small \thepage}
\renewcommand{\headrulewidth}{0pt}
\renewcommand{\footrulewidth}{0pt}}
\pagestyle{newstyle}

% geometry of the page
\usepackage[top=1in, bottom=1in, left=1in, right=1in]{geometry}

% paragraph spacing
\setlength{\parindent}{0pt}
\setlength{\parskip}{2ex plus 0.4ex minus 0.2ex}

% useful packages
\usepackage{natbib}
\usepackage{epsfig}
\usepackage{url}
\usepackage{bm}


\begin{document}

\begin{center}
  \Large \textbf{Week 6 Response: \textit{Imbens \& Rubin, Chapter 12}} \\
  \vspace{0.1in}
  \normalsize Natalie Carlson \\
  \today
\end{center}

I found this chapter delightfully clear, although I am starting to get a little bogged down in the overlapping terminology we've encountered so far. Being grounded in the paradigm of economics, I am familiar with discussions of \textit{exogeneity} and \textit{endogeneity}. Morgan \& Winship brought in the term \textit{ignorability}. And now we have \textit{unconfoundedness} -- if I'm not completely confused (possible), these terms all address the same general idea; that is, they refer to the central problem in any discussion of causal inference with observational data. The example given of employment training programs is illustrative of this problem and rang very familiar. It is almost impossible in situations like this to account for every unobserved variable that might negate unconfoundedness -- and reviewers excel in positing inventive ways that the assumption might be violated -- which is why social scientists increasingly tend to resort to randomized experiments or search for natural experiments to exploit.

This chapter introduced a few new terms to me as well, one being the notion of a balancing score (very intuitive in its relationship to the propensity score) and the other being the semiparametric efficiency bound. I had not encountered the latter before and I'm not sure I completely understand the context in which it would be used, particularly because I did not see it referred to again in their helpful ending discussion on strategies for estimation.

\end{document}
