\documentclass[12pt]{article}

% fonts
\usepackage[scaled=0.92]{helvet}   % set Helvetica as the sans-serif font
\renewcommand{\rmdefault}{ptm}     % set Times as the default text font

% dmb: not mandatory, but i recommend you use mtpro for math fonts.
% there is a free version called mtprolite.

% \usepackage[amssymbols,subscriptcorrection,slantedGreek,nofontinfo]{mtpro2}

\usepackage[T1]{fontenc}
\usepackage{amsmath}
\usepackage{amsfonts}

% page numbers
\usepackage{fancyhdr}
\fancypagestyle{newstyle}{
\fancyhf{} % clear all header and footer fields
\fancyfoot[R]{\vspace{0.1in} \small \thepage}
\renewcommand{\headrulewidth}{0pt}
\renewcommand{\footrulewidth}{0pt}}
\pagestyle{newstyle}

% geometry of the page
\usepackage[top=1in, bottom=1in, left=1in, right=1in]{geometry}

% paragraph spacing
\setlength{\parindent}{0pt}
\setlength{\parskip}{2ex plus 0.4ex minus 0.2ex}

% useful packages
\usepackage{natbib}
\usepackage{epsfig}
\usepackage{url}
\usepackage{bm}


\begin{document}

\begin{center}
  \Large \textbf{Week 1 Response: \textit{Morgan \& Winship 2015, Ch. 2}} \\
  \vspace{0.1in}
  \normalsize Natalie Carlson \\
  \today
\end{center}

This chapter -- as well as the seminar yesterday -- took me back to a lot of concepts that I am generally quite comfortable with and forced me to examine their foundations in detail. Although it was occasionally painful, I found it useful to grapple with not just the statistical meaning of a causal statement, but the underlying metaphysics as well. 

I appreciated the thoroughness of the Morgan \& Winship chapter, although I thought it occasionally descended into unnecessarily wordy academic-\textit{ese}. An example: early on in the chapter they were discussing when it is useful to break down a treatment into components, in order to more narrowly identify the causal mechanism, i.e. the ``black box'' problem. Their take: ``Accordingly, we see little value in making general arguments about when such decomposition is feasible because of the inherent separability of the productive capacities attached to particular constitutive features or is infeasible because of the deeply entangled complementarities among them.'' Yikes! Just say that in some cases it is easier to isolate specific components of the treatment than in others. 

That is a small complaint, however. I enjoyed the examples and found it useful to review the math behind the naive estimator, and the assumptions necessary for it to equal the ATE. I started my research career managing RCTs for a development economist, so was happily ensconced in the world of perfect causal identification for some time before having to confront the weaknesses of observational data. I am tempted to suggest we devote our resources to accessing the ``science fiction'' world in which we can observe the counterfactual -- but barring that possibility, I'm looking forward to studying more earthly methods.  



\end{document}
