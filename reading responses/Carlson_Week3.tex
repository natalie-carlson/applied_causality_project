\documentclass[12pt]{article}

% fonts
\usepackage[scaled=0.92]{helvet}   % set Helvetica as the sans-serif font
\renewcommand{\rmdefault}{ptm}     % set Times as the default text font

% dmb: not mandatory, but i recommend you use mtpro for math fonts.
% there is a free version called mtprolite.

% \usepackage[amssymbols,subscriptcorrection,slantedGreek,nofontinfo]{mtpro2}

\usepackage[T1]{fontenc}
\usepackage{amsmath}
\usepackage{amsfonts}

% page numbers
\usepackage{fancyhdr}
\fancypagestyle{newstyle}{
\fancyhf{} % clear all header and footer fields
\fancyfoot[R]{\vspace{0.1in} \small \thepage}
\renewcommand{\headrulewidth}{0pt}
\renewcommand{\footrulewidth}{0pt}}
\pagestyle{newstyle}

% geometry of the page
\usepackage[top=1in, bottom=1in, left=1in, right=1in]{geometry}

% paragraph spacing
\setlength{\parindent}{0pt}
\setlength{\parskip}{2ex plus 0.4ex minus 0.2ex}

% useful packages
\usepackage{natbib}
\usepackage{epsfig}
\usepackage{url}
\usepackage{bm}


\begin{document}

\begin{center}
  \Large \textbf{Week 3 Response: \textit{Morgan \& Winship 2015, Ch. 3}} \\
  \vspace{0.1in}
  \normalsize Natalie Carlson \\
  \today
\end{center}

I will call this reading response a combined response to the Morgan \& Winship and the Pearl reading, although I chose to highlight the former because the Pearl reading was considerably more painful to work through (with all due respect to the comprehensiveness and utility of the theory). As we worked through most of this material last week in class, nothing in these readings was terribly suprising. One of the things that has continued to bother me, though, is our discussion of how to extend the causal graphs to the framework of plate diagrams. I'm not sure I understand the inclusion of the error term in the plates and the parameterization outside the plates of the errors (I think it was theta). Was \textit{theta} just a marker for some kind of assumed error distribution across the sample (Gaussian under OLS assumptions, etc.)? 

When I took graphical models a couple years ago, I think I compartmentalized the material in a totally separate part of my brain from all the econometrics I've studied. This material is forcing me to integrate the frameworks, which is difficult, but a very good exercise.

\end{document}
