\documentclass[12pt]{article}

\input{preamble}

\begin{document}

\begin{center}
  \Large \textbf{Week 2 Response: \textit{Morgan \& Winship 2015, Ch. 4}} \\
  \vspace{0.1in}
  \normalsize Natalie Carlson \\
  \today
\end{center}

I have not encountered the ``back door criterion'' before, but it was extraordinarily satisfying to work my way through it, because it helped me to actually understand the Bayes' Ball algorithm. Previously, to be honest, I had mostly memorized Bayes' Ball as an abstract set of rules, with only a vague understanding of the reasoning behind them. I had a tenuous grasp on the ``aliens/broken watch'' notion, but somehow framing things in terms of a collider variable really made it click. I found the college admissions example especially helpful. 

In applying these concepts to my own problem, I think I may still struggle, however. One of the problems is that nearly all of the edges in any graph I attempt with my data could potentially be bidirectional. I also think one of the main conclusions of the back-door criterion -- that one does not need to condition on all the variables in a causal chain to account for omitted variable bias -- is something that is not generally followed in the social sciences, where the general ethos typically resembles something more like ``the more control variables, the better.'' As such, in practice I think you tend to find a lot of violations of Condition 2. Of course, in the social sciences, the goal tends to be identifying the existence of a causal effect, rather than getting a precise estimate of the ATE, so any conditioning variable that will bias the estimated ATE downwards is considered a conservative action and will generally not be challenged. 

\end{document}
