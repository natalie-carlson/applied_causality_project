\documentclass[12pt]{article}

\input{preamble}

\begin{document}

\begin{center}
  \Large \textbf{Week 4 Response: \textit{Pearl 2009, Chapter 3}} \\
  \vspace{0.1in}
  \normalsize Natalie Carlson \\
  \today
\end{center}

I actually enjoyed this reading much more than the previous Pearl chapter we had tackled. In particular, I found framing the \textit{do}(.) operator as a variable to be very appealing, mainly because defining the conditional probability of the causal variable relative to the intervention function -- including for the \textit{idle} nonintervention condition -- adds some useful clarity. I'm not sure I understand the usefulness of the front-door criterion, however. This struck me as somewhat trivial. Of course if there is a concomitant in the direct causal path of interest it can be used to estimate the causal effect. Perhaps I am misunderstanding?

I enjoyed both Shalizi chapters as well, and Chapter 24 in particular began to introduce many of the solutions to causal inference problems I am accustomed to from econometrics. All of these are appealing, but for my personal project many of the variables of interest are measuring quite fuzzy, interrelated constructs, so it's hard to imagine a propensity matching or IV approach that wouldn't encounter a lot of resistance. Personally, I love a good clean instrument. One of my favorite examples used the channel position of various news stations to estimate the persuasive effort of viewing slanted news coverage (people tend to watch more the channels with lower numbers) [1]. So clever!


[1] \textit{Martin, Gregory J., and Ali Yurukoglu. Bias in cable news: Real effects and polarization. National Bureau of Economic Research, 2014.}


\end{document}
